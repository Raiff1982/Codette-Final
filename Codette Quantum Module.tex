\documentclass[11pt,a4paper]{article}
\usepackage{graphicx}
\usepackage{hyperref}
\usepackage{amsmath}
\usepackage{booktabs}
\title{Citizen-Science Quantum and Chaos Simulations Orchestrated by the Codette AI Suite}
\author{Jonathan Harrison\\Raiffs Bits LLC\\\texttt{jonathan.harrison@example.com}\\ORCID: 0009-0003-7005-8187}
\date{May 2025}

\begin{document}
\maketitle

\begin{abstract}
We present a modular citizen-science framework for conducting distributed quantum and chaos simulations on commodity hardware, augmented by AI-driven analysis and meta-commentary. Our Python-based Codette AI Suite orchestrates multi-core trials seeded with live NASA exoplanet data, wraps each run in encrypted “cocoons,” and applies recursive reasoning across multiple perspectives. Downstream analyses include neural activation classification, dream-state transformations, and clustering in 3D feature space, culminating in an interactive timeline animation and a transparent artifact bundle. This approach democratizes quantum experimentation, providing reproducible pipelines and audit-ready documentation for both scientific and educational communities.
\end{abstract}

\section{Introduction}
% stub – introduce the motivation and background

\section{Methods}
\subsection{Quantum and Chaos Simulation}
% stub – outline scripts quantum\_cosmic\_multicore.py

\subsection{Cocoon Data Wrapping}
% stub – describe cognition\_cocooner.py

\subsection{AI-Driven Meta-Analysis}
% stub – list codette\_quantum\_multicore2.py, meta\_3d, timeline

\section{Results}
% stub – placeholder for Meta Reflection Table and figure inserts

\section{Discussion}
% stub – interpret stability, variability, and AI insights

\section{Conclusion}
% stub – summarize contributions and future work

\section*{Availability}
All code and artifacts: \url{https://github.com/Raiff1982/codette-quantum}.

\bibliographystyle{plain}
\begin{thebibliography}{9}
\bibitem{nasaExoAPI} NASA Exoplanet Archive, \url{https://exoplanetarchive.ipac.caltech.edu/}.
\end{thebibliography}

\end{document}

